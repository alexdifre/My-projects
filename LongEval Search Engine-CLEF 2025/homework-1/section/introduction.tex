\section{Introduction}
\label{sec:introduction}

The amount of information available online has been rapidly increasing due to the internet's fast growth, making it difficult to find reliable and pertinent information. Search engines are essential in this situation as they allow users to quickly and easily locate accurate information. Information retrieval systems are now essential in almost all the fields. Our study investigates the complexities of information retrieval and how it can be enhanced, as well as how well they can adjust to the changing digital environment. 

The goal of this research is to create a reliable retrieval system that can handle the ever-changing nature of the internet. We do this by using the Longeval Websearch collection \cite{longeval2025}, an extensive dataset of online pages, topics, and user interactions. As a result of this effort, our understanding of information retrieval systems in the context of search engines has greatly improved. In particular, we focused on improving query and document processing methods to maximise ranking outcomes and provide users with the most relevant information.

The paper is organized as follows: Section~\ref{sec:methodology} describes our approach; Section~\ref{sec:setup} explains our experimental setup; Section~\ref{sec:results} discusses our main findings; finally, Section~\ref{sec:conclusion} draws some conclusions and outlooks for future work.

\subsection{Related Work}
To begin addressing the project requirements, we conducted a thorough analysis of the specifications and file formats necessary for implementation. We relied on materials provided by the CLEF LongEval \cite{longeval2025} website and the accompanying 2023 documentation \cite{alkhalifa2023clef} to identify the essential elements needed to organize our workflow effectively. Guided by the tutoring sessions from the SearchEngines course at the University of Padua, we proceeded with the development of the key components—indexer, analyzer, and searcher—following the framework introduced and discussed in class. For implementing the re-ranker, we employed the API offered and documented by Cohere \cite{cohereAbout2025}.